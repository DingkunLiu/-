%!TEX program=xelatex
\documentclass{article}
\usepackage{ctex}
\usepackage{amsmath}
\usepackage{amssymb}
\usepackage{amsthm}
\usepackage{booktabs}
\usepackage{array}
\usepackage{graphicx}
\usepackage{hyperref}

\newtheorem{definition}{定义}[section]
\newtheorem{theorem}{定理}[section]
\newtheorem{lemma}{引理}[section]
\newtheorem{example}{例}

\makeatletter % `@' now normal "letter"
\@addtoreset{equation}{section}
\makeatother  % `@' is restored as "non-letter"
\renewcommand\theequation{\oldstylenums{\thesection}%
                   .\oldstylenums{\arabic{equation}}}

\begin{document}
\title{第五章\ 常微分方程初值问题的数值解法}
\author{}
\date{}
\maketitle

常微分方程的一般形式:
\begin{equation}
    \left\{
        \begin{array}{lr}
            \frac{dy}{dx} = f(x, y) \\
            y(x_0) = y_0
        \end{array}
        \right. 
\label{eq:diff}
\end{equation}

(定解条件)对于问题\ref{eq:diff},如果函数$f(x,y)$关于y满足Lipschitz条件,即$\exists L$使得$|f(x,y)-f(x,\overline{y})|\le L|y-\overline{y}|$成立
则由微分方程理论可知,初值问题\ref{eq:diff}的解存在且唯一。

首先,将自变量x的区间进行离散,i.e. $x_n = x_{n-1} + h_n$,若等步长,则$x_n = x_0 + nh$

\section{Euler方法}
\subsection{向前Euler方法}
用$y_n$表示$y(x_n)$的近似解,i.e. $y_n \approx y(x_n)$
由\ref{eq:diff}显然有$\frac{dy}{dx}|_{x=x_n} = f(x_n, y(x_n)) = y^{'}(x_n)$,
而$\frac{dy}{dx}|_{x=x_n} \approx \frac{y(x_{n+1}) - y(x_n)}{h}$,i.e. $\frac{y(x_{n+1}) - y(x_n)}{h} \approx f(x_n, y(x_n))$
$\Rightarrow y(x_{n+1}) \approx y(x_n) + hf(x_n, y(x_n))$ ,若用$y_n\approx y(x_n), y_{n+1}\approx y(x_{n+1})$,则上式可写为:
\begin{equation}
    y_{n+1}=y_n+hf(x_n, y_n), n=0,1,2,\dots 
    \label{eq:euler}
\end{equation}
向前Euler公式,显式方法。

泰勒公式展开法推导
\begin{equation*}
    \begin{split}
        &y(x_{n+1}) = y(x_n) + hy^{'}(x_n) +\frac{h^2}{2}y^{{'}{'}}(\xi) \\
        \Rightarrow& y(x_{n+1}) \approx y(x_n)+hf(x_n, y(x_n)) \\
        \Rightarrow& 
        y_{n+1} = y_n +hf(x_n, y_n)
    \end{split}
\end{equation*}

积分方法推导:
由\ref{eq:diff}可得:
\begin{equation*}
    \begin{array}{lr}
        \int^{x_{n+1}}_{x_n} \frac{dy}{dx}dx = \int^{x_{n+1}}_{x_n} f(x,y)dx=y(x_{n+1}) - y(x_n) \approx \\
        f(x_n, y(x_n))(x_{n+1} - x_n) = hf(x_n, y(x_n)) 
    \end{array}
\end{equation*}

\subsection{向后Euler公式,梯形公式和改进Euler公式}
向后Euler公式,隐式公式:
\begin{equation}
    \begin{array}{lr}
        y(x_{n+1}) \approx y(x_n) + hf(x_{n+1}, y(x_{n+1})) \Rightarrow \\
        y_{n+1} = y_n +hf(x_{n+1}, y_{n+1})
    \end{array}
    \label{eq:eulerback}
\end{equation}

梯形公式(隐式公式),当$f(x,y)$关于y是线性,则可以直接求出$y_{n+1}$:
\begin{equation}
    \begin{array}{lr}
        f(x, y(x)) \approx \frac{f(x_n, y(x_n))+f(x_{n+1}, y(x_{n+1}))}{2} \\
        y(x_{n+1}) \approx y(x_n) + h\frac{f(x_n, y(x_n))+f(x_{n+1}, y(x_{n+1}))}{2}, n=0,1,2,\dots
    \end{array}
    \label{eq:eulermid}
\end{equation}

隐式方法的稳定性往往较好(可取较大步长),但对于非线性问题,方程难解。

以梯形公式\ref{eq:eulermid}为例说明如何而求$y_{n+1}$:
\begin{equation*}
    \begin{array}{lr}
        y^{(0)}_{n+1} = y_n + hf(x_n, y_n) \\
        y^{(k+1)}_{n+1} = y_n + h\frac{f(x_n, y_n) + f(x_{n+1}, y^{(k)}_{n+1})}{2}, k=0,1,2,\dots
    \end{array}
\end{equation*}

若迭代一次即终止,则有如下公式:
\begin{equation*}
    \begin{array}{lr}
        \overline{y}_{n+1} = y_n + hf(x_n, y_n) \\
        y_{n+1} = y_n +\frac{h}{2}[f(x_n, y_n)+f(x_{n+1}, \overline{y}_{n+1})], n=0,1,2,\dots
    \end{array}
\end{equation*}
整理可得改进Euler公式:
\begin{equation}
    y_{n+1} = y_n + \frac{h}{2}[f(x_n, y_n)+ f(x_{n+1}, y_n + hf(x_n, y_n))]
    \label{eq:eulerimprov}
\end{equation}

\subsection{单步法的截断误差和阶}
所有单步法均可写成:
$$y_{n+1} = y_n + h\phi(x_n, y_n, y_{n+1}, h) $$
其中$\phi$与f有关,$\phi$称为增量函数

若方法显式则:
\begin{equation}
    y_{n+1}=y_n+h\phi(x_n, y_n, h) 
    \label{eq:explicit}
\end{equation}

称$e_{n+1} = y(x_{n+1})-y_{n+1} $为在$x_{n+1}$点的整体截断误差

\begin{definition}
    设$y(x)$是初值问题\ref{eq:diff}的准确解。称
    \begin{equation}
        T_{n+1}(x) = y(x_{n+1})-y(x_n)-h\phi(x_n, y(x_n), h) 
    \end{equation}
    为显式单步法\ref{eq:explicit}在$x_{n+1}$处的局部截断误差。
\end{definition}

\begin{definition}
    若$T_{n+1} = O(h^{P+1})$,则称此方法具有P阶精度,若$T_{n+1}=\phi^*(x_n, y(x_n))h^{P+1}+O(h^{P+2})$,则称
    $\phi^*(x_n, y(x_n))h^{P+1}$为该方法的局部截断误差主项。
\end{definition}

对于单步隐式方法$T_{n+1}=y(x_{n+1}) -y(x_n)-h\phi(x_n, y(x_n), y(x_{n+1}), h) $

例:求梯形公式的局部截断误差主项和阶
解:
\begin{equation*}
    \begin{split}
        &T_{n+1} = y(x_{n+1})-y(x_n)-h\frac{f(x_n, y(x_n))+f(x_{n+1},y(x_{n+1}))}{2} \\
        =&y(x_n+h)-y(x_n)-\frac{h}{2}[y^{'}(x_n)+ y^{'}(x_{n}+h)] \\
        =&-\frac{h^3}{12}y^{{'}{'}{'}}(x_n)+O(h^4)
    \end{split}
\end{equation*}
$\therefore$梯形公式的局部截断误差为$-\frac{h^3}{12}y^{{'}{'}{'}}(x_n)+O(h^4)$,主项为:$-\frac{h^3}{12}y^{{'}{'}{'}}(x_n)$,阶:2阶。

思考题:改进Euler方法的$T_{n+1}=?$主项为?阶?
(多元函数求2阶导)

\begin{proof}
    \begin{equation*}
        \begin{split}
            &y(x_{n+1})=y(x_n)+hf+\frac{h^2}{2}(f_x+ff_y)+ \\
            &\quad \frac{h^3}{6}(f_{xx} + f_{xy}f + ff_{yx}+f^2f_{yy}+f_yf_x+f^2_yf) + O(h^4)
        \end{split}
    \end{equation*}
    \begin{equation*}
        \begin{split}
            &\overline{y} = y(x_n) + \frac{h}{2}f + \frac{h}{2}(f+ h(f_x+ff_y) \\
            &\quad \frac{h^2}{2}(f_{xx} + f_{xy}f + ff_{yx}+f^2f_{yy}+f_yf_x+f^2_yf) + O(h^3))
        \end{split}
    \end{equation*}
    \begin{align*}
        &y(x_{n+1}) - \overline{y} \\
        =& -\frac{h^3}{12}(f_{xx} + f_{xy}f + ff_{yx}+f^2f_{yy}+f_yf_x+f^2_yf) + O(h^4)
    \end{align*}
\end{proof}

\section{Runge-kutta方法}
\subsection{R-K法基本思想}
向前Euler(1阶):
$$
\left\{
\begin{array}{lr}
    y_{n+1} = y_n+hk_1\\
    k_1=f(x_n, y_n)
\end{array}\right.
$$

梯形公式(2阶):
$$
\left\{
\begin{array}{lr}
    y_{n+1} = y_n + h(\frac{1}{2}k_1+\frac{1}{2}k_2)\\
    k_1 = f(x_n, y_n)\\
    k_2 = f(x_{n+1}, y_{n+1})
\end{array}\right.
$$

取更多点斜率加权平均,构造更高精度的数值方法。

\subsection{R-K方法的构造}
假设P级R-K方法的公式为:
\begin{equation}
\begin{array}{lr}
    y_{n+1} = y_n +h\sum^P_{i=1}c_ik_i, &k_1=f(x_n,y_n)\\
    k_i = f(x_n+a_ih, y_n+h\sum^{i-1}_{j=1}b_{ij}k_j), & i=2,3,\dots, P 
\end{array}
\end{equation}

以$P=2$为例,构造R-K方法:
$$
\begin{array}{lr}
    y_{n+1} = y_n + h(c_1k_1+c_2k_2), k_1=f(x_n, y_n)\\
    k_2 = f(x_n+a_2h, y_n+hb_{21}k_1)
\end{array}
$$

\begin{equation*}
    \begin{array}{lr}
        T_{n+1}=y(x_{n+1})-y(x_n)-h\{c_1f(x_n, y(x_n)) \\
        \quad \quad +c_2f[x_n+a_2h, y(x_n+b_{21}hf(x_n, y(x_n)))]\}
    \end{array}
\end{equation*}

构造麻烦的地方在于$y(x_n+b_{21}f(x_n, y(x_n))$无法写成y的形式,使用2元函数泰勒展开。
\begin{equation*}
    \begin{split}
        &y(x_{n+1})=[y+hy^{'}+\frac{h^2}{2}y^{{'}{'}}+\frac{h^3}{6}y^{{'}{'}{'}}+O(h^4)]|_{x=x_n}\\
        =&y+hf+\frac{h^2}{2}(f_x+f_yf)+\frac{h^3}{6}() \\
        &\overline{y}_{n+1} = y + h\{c_1f + c_2[f+a_2hf_x+hb_{21}ff_y]\} + O(h^3)
    \end{split}
\end{equation*}

将$y(x_{n+1})$与$\overline{y}_{n+1}$中前三项对应相等即可得:
$$
\left\{
\begin{array}{lr}
    c_1+c_2 = 1 \\
    c_2a_2 = \frac{1}{2} \\
    c_2b_{21} = \frac{1}{2}
\end{array}\right.
$$
取$c_2=1$则$c_1=0,a_2=b_{21}=\frac{1}{2}$,
$$
\left\{
\begin{array}{lr}
    k_1 = f(x_n, y_n) \\
    k_2 = f(x_n+\frac{1}{2}h, y_n+\frac{1}{2}hk_1)
\end{array}\right.
$$

\begin{equation}
    \left\{
        \begin{array}{lr}
            y_{n+1} = y_n+hk_2 \\
            y_{n+1} = y_n+hf(x_n+\frac{1}{2}h, y_n+\frac{1}{2}hf(x_n, y_n))
        \end{array}
        \right.
\end{equation}

如果取$c_1=c_2=\frac{1}{2}, a_2=b_{21}=1$即可得改进Euler公式\ref{eq:eulerimprov}。

4阶R-K方法:
\begin{equation}
    \left\{
    \begin{array}{lr}
        y_{n+1} = y_n +\frac{h}{6}[k_1+2k_2+2k_3+k_4], k_1=f(x_n, y_n) \\
        k_2 = f(x_n+\frac{1}{2}h, y_n+\frac{h}{2}k_1), k_3=f(x_n+\frac{1}{2}h, y_n+\frac{h}{2}k_2) \\
        k_4 = f(x_n+h, y_n+hk_3)
    \end{array}\right.
\end{equation}

\section{单步法基本理论}
\begin{definition}
    对于所有初值问题(\ref{eq:diff})(假设f满足Lipschitz条件)的一种单步法:
    \begin{equation}
        y_{n+1}=y_n+h\phi(x_n,y_n,h)
        \label{eq:3.1}
    \end{equation}
    产生的近似解,若对任意固定的$x_n=x_0+nh$,均有$\lim_{h\rightarrow 0}y_n=y(x_n) $则称方法\ref{eq:3.1}是收敛的。
\end{definition}

\begin{definition}
    若方法\ref{eq:3.1}的增量函数$\phi$满足条件:
    \begin{equation*}
        \phi(x,y,0)=f(x,y)
    \end{equation*}
    则称方法\ref{eq:3.1}与初值问题\ref{eq:diff}相容。
\end{definition}

\paragraph{稳定性}
以向前Euler方法\ref{eq:euler}为例,$y_{n+1}=y_n+hf(x_n,y_n) $假设$\bar{y}_n=y_n+\varepsilon_n$($\varepsilon_n$为计算误差)
,记$\varepsilon_{n+1}=\bar{y}_{n+1}-y_{n+1} , \bar{y}_{n+1}=\bar{y}_n+hf(x_n, \bar{y}_n) $,
上述两式相减得:
\begin{equation*}
    \begin{split}
        \varepsilon_{n+1} &= \varepsilon_n +h[f(x_n, \bar{y}_n)-f(x_n,y_n)] \\
        &= \varepsilon + h(f^{'}_y(x_n, \eta)(\bar{y_n}-y_n)) \\
        &=[1+hf^{'}_y(x_n, \eta)]\varepsilon_n
    \end{split}
\end{equation*}
$\therefore \ if. \ |1+hf^{'}_y(x_n, \eta)|\le 1 \Rightarrow |\varepsilon_{n+1} |\le \varepsilon_n$
则此时向前Euler方法稳定,否则不稳定。\\
由此可得,若$f^{'}_y>0 $,则称方程\ref{eq:diff}具有先天性不稳定性。

将数值方法应用于如下试验方程:
$$y^{'}=\lambda y $$
在此种情况易证
\begin{equation}
    y_{n+1}=E(\lambda h)y_n 
    \label{eq:*}
\end{equation}

\begin{definition}
    如果\ref{eq:*}式中,$|E(\lambda h)|<1$,则称方法\ref{eq:3.1}是绝对稳定的,在复平面上复变量$\lambda h$满足$|E(\lambda h)|<1$的区域,称此方法\ref{eq:3.1}的绝对稳定性区域,它与实轴的交称为绝对稳定区间。
\end{definition}

\begin{example}
    已知求解初值问题$y^{'}=f(x,y), y(x_0)=y_0$的显式单步法
    $$y_{n+1}=y_n+\frac{1}{4}h[f(x_n,y_n)+3f(x_n+\frac{2}{3}h,y_n+\frac{2}{3}hf(x_n,y_n))] $$
    \begin{enumerate}
        \item 求其局部截断误差和精度阶数
        \item 求其绝对稳定性区域和绝对稳定性区间
    \end{enumerate}

    \paragraph{解}$T_{n+1}=y(x_{n+1})-\widetilde{y}_{n+1} $
    $$y(x_{n+1})=y(x_n+h)=y(x_n)+hy^{'}(x_n)+\frac{h^2}{2}y^{(2)}(x_n)+\frac{h^3}{6}y^{(3)}(x_n)+O(h^4) $$
    而$y^{'}=f_x+f_yf, y^{{'}{'}}=f_{xx}+2ff_{xy}+f^2f_{yy}+f_y(f_x+ff_y) $\\
    \begin{equation*}
        \begin{split}
            \widetilde{y}_{n+1}&=y(x_n) + \frac{h}{4}\{f+3[f+f_x\frac{2}{3}h+f_y\frac{2}{3}hf+\\
            &\frac{1}{2}(f_{xx}(\frac{2}{3}h)^2+2f_{xy}\frac{2}{3}hf+f_{yy}(\frac{2}{3}hf)^2)]+O(h^3) \}
        \end{split}
    \end{equation*}
    \begin{equation*}
        \begin{split}
            T_{n+1} &= y(x_{n+1})-\widetilde{y}_{n+1}\\
            &=\frac{h^3}{6}f_y(x_n, y(x_n))[f_x(x_n,y(x_n))+f(x_n,y(x_n)f_y(x_n,y(x_n)))]+O(h^4)
        \end{split}
    \end{equation*}
    所以精度为2阶

    (2)将该方法应用于试验方程$y^{'}=\lambda y$得
    \begin{equation*}
        \begin{split}
            y_{n+1}&=y_n+\frac{1}{4}h[\lambda y_n+3\lambda (y_n+\frac{2}{3}hf(x_n, y_n))] \\
            &=[1+\lambda h+\frac{1}{2}(\lambda h)^2]y_n
        \end{split}
    \end{equation*}
    $\therefore$绝对稳定性区域为$|1+\lambda h + \frac{1}{2}(\lambda h)^2|<1$\\
    由上述不等式$\Rightarrow \lambda h(\frac{1}{2}\lambda h+1)<0 \Rightarrow \lambda h>-2 $\\
    $\therefore $绝对稳定性区间为$(-2,0)$($\lambda$必须为负,否则方程具有先天性不稳定性。)
\end{example}

\begin{example}
    将改进的Euler方法应用于求解初值问题$y^{'}=-20y, y(x_0)=y_0 $,试问步长h满足什么条件时才能得证该方法的绝对稳定性。
    
    \paragraph{解}$\because y_{n+1}=y_n+\frac{h}{2}[f(x_n, y_n)+f(x_n+h, y_n+hf(x_n,y_n))] $\\
    将上述方法应用于$y^{'}=-20y$得:
    $$
    \begin{array}{lr}
        y_{n+1}=y_n+\frac{h}{2}[-20y_n+-20(y_n+hf(x_n, y_n))]\\
        =y_n+\frac{h}{2}[-20y_n-20(y_n-20hy_n)] \\
        =[1-20h+\frac{1}{2}(20h)^2]y_n
    \end{array}
    $$
    $\therefore E(\lambda h)=1-20h+\frac{1}{2}(20h)^2, \therefore |E(\lambda h)|<1 $得$h <0.1$
\end{example}

\end{document}