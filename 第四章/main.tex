%!TEX program=xelatex
\documentclass[a4paper]{article}

\usepackage{amsmath}
\usepackage{amssymb}
\usepackage{amsthm}
\usepackage{ctex}
\usepackage{hyperref}

\newtheorem{definition}{定义}[section]
\newtheorem{theorem}{定理}[section]
\newtheorem{lemma}{引理}[section]
\newtheorem{example}{例}[section]

\makeatletter % `@' now normal "letter"
\@addtoreset{equation}{section}
\makeatother  % `@' is restored as "non-letter"
\renewcommand\theequation{\oldstylenums{\thesection}%
                   .\oldstylenums{\arabic{equation}}}

\DeclareMathOperator{\spn}{span}

\begin{document}
\title{第四章\ 数值逼近与数值积分}
\author{}
\date{}
\maketitle

\section{正交多项式}
\begin{definition}
    内积:(两个函数内积)设$f,g \in [a,b], \rho(x)$为权函数,称$(f, g) = \int^b_a\rho(x)f(x)g(x)dx$为函数f和g在$[a,b]$上带权$\rho$的内积。
\end{definition}

\begin{definition}
    正交:设$f,g\in C[a,b]$,$\rho$为权函数,若$(f,g)=0$,则称f与g在$[a,b]$上带权$\rho$正交。
\end{definition}

\begin{definition}
    正交多项式:设$\phi_n$是$[a,b]$上首项系数$a_n\neq 0$的n次多项式,$\rho(x)$为$[a,b]$上权函数,若多项式序列$\phi_0, \phi_1, \dots, \phi_n$满足:
    $$(\phi_i, \phi_j)=\int^b_a\rho(x)\phi_i(x)\phi_j(x)dx = \left\{\begin{array}{lr}
        0, i\neq j \\ A_j > 0, i=j
    \end{array}\right.$$则称$\phi_0, \phi_1, \dots, \phi_n$为$[a,b]$上带权$\rho(x)$正交,并称$\phi_n(x)$为$[a,b]$上带$\rho(x)$的n次正交多项式。
\end{definition}

\subsection{Gram-Schmidt方法}
用Gram-Schmidt正交化方法构造$[a,b]$上带权$\rho(x)$的正交多项式\textbf{(重点)}:
\begin{align*}
    &\phi_0(x) = 1\\
    &\phi_1(x) = x - \alpha_1 \\
    &\phi_k(x) = (x-\alpha_k)\phi_{k-1}(x) - \beta_k\phi_{k-2}(x), k=2,3,\dots
\end{align*}
其中:$\alpha_k = \frac{(x\phi_{k-1}, \phi_{k-1})}{(\phi_{k-1}, \phi_{k-1})}, \beta_k = \frac{(\phi_{k-1}, \phi_{k-1})}{(\phi_{k-2}, \phi_{k-2})}$

\subsection{正交多项式性质}
\begin{enumerate}
    \item $\phi_k(x)$是最高次数项系数为1的多项式。
    \item $\phi_0, \phi_1, \dots, \phi_n$在$[a,b]$上带权$\rho(x)$正交
    \item 区间$[a,b]$上关于权函数$\rho(x)$,正交多项式$\phi_0, \phi_1, \dots, \phi_n$必定\textit{线性无关}。
    \item 任意k次多项式均可表示为前k+1个$\phi_0, \phi_1, \dots, \phi_k$的线性组合$P_k(x) = c_o\phi_0 + c_1\phi_1 + \dots + c_k\phi_k$
\end{enumerate}

\begin{theorem}
    设$\{\phi_k, k\ge 0\}$是$[a,b]$上带权$\rho(x)$的正交多项式序列,则n次多项式$\phi_n(x)$在$(a,b)$内恰有n个不同的实零点($\phi_n(x_i)=0, i=1,2, \dots, n$,即$\phi_n(x) = a_n(x-x_1)(x-x_2)\cdots(x-x_n)$)。
\end{theorem}

\subsection{常见的多项式}
\subsubsection{Legendre多项式}
表达式:
\begin{equation}
    \left\{
        \begin{array}{lr}
            P_0(x) = 1 \\
            P_n(x) = \frac{1}{2^nn!}\frac{d^n}{dx^n}\{(x^2-1)^n\}
        \end{array}
        \right.
\end{equation}

性质:
\begin{enumerate}
    \item 正交性: 
    $$\int^1_{-1}P_n(x)P_m(x)dx = \left\{
        \begin{array}{lr}
            0, n\neq m \\
            \frac{2}{2n+1}, n = m
        \end{array}
        \right.$$
        i.e. Legendre多项式序列$P_0, P_1, \dots, P_n$是$[-1,1]$上带权$\rho(x)=1$的正交多项式序列。
    \item 奇偶性:$P_n(-x) = (-1)^nP_n(x)$
    \item 递推关系:$(n+1)P_{n+1}(x) = (2n+1)xP_n(x) - nP_{n-1}(x), n=0,1,2,\dots$
    \item $P_n(x)$在$(-1,1)$内有n个不同的实零点$P_n(x) = a_n(x-x_1)(x-x_2)\cdots(x-x_n)$
\end{enumerate}

Legendre多项式首项系数:$a_n = \frac{(2n)!}{2^n(n!)^2}$,从表达式易知。

\subsubsection{切比雪夫(Chebyshev)多项式}
对于$x\in [-1, 1]$,称n次多项式:
\begin{equation}
    T_n(x) = cos(n arc cos x), n = 0, 1, 2, \dots 
\end{equation}
为chebyshev多项式。

性质:
\begin{enumerate}
    \item 正交性:
    $$\int^1_{-1} \frac{T_n(x)T_m(x)}{\sqrt{1-x^2}}dx = \left\{
        \begin{array}{lr}
            0, n \neq m \\
            \frac{\pi}{2}, n=m \neq 0 \\
            \pi, n=m=0
        \end{array}
        \right. $$
        i.e. Chebyshev多项式在$[-1,1]$上带权$\rho(x) = \frac{1}{\sqrt{1-x^2}}$正交
    \item 递推关系:$T_{n+1}(x) =2xT_n(x)-T_{n-1}(x), n=1,2,\dots $
    \item 奇偶性:$T_n(-x) = (-1)^nT_n(x) $
    \item $T_n(x)$在$(-1,1)$上有n个不同的实零点;$x_k = cos\frac{(2k-1)\pi}{2n}, k=1,2,\dots,n$
    \item $T_n$的首项系数是$a_n = 2^{n-1},n=1,2,\dots$
    \item 任意n次多项式均可表示成$T_0, T_1, \dots, T_n$的线性组合:$P_n(x) = c_oT_0 + c_1T_1 + \dots + c_nT_n$      
\end{enumerate}

由性质1和性质6可知:
$$\int^1_{-1}\frac{P_k(x)T_n(x)}{\sqrt{1-x^2}}dx = 0, k < n$$

\section{最小二乘拟合}

\subsection{线性最小二乘拟合}
已知一组数据$(x_i, f_i), i=0,1,\dots, m$,要求在函数类$\Phi = \spn\{\phi_0, \phi_1, \dots, \phi_n\}$中找到一个函数$S(x)$
使得
\begin{equation}
    \sum^m_{j=0}\rho(x_j)[S^*(x_j)-f_j]^2=\min_{S(x)\in\Phi}\sum^m_{j=0}\rho(x_j)[S(x_j)-f_j]^2 
\end{equation}
其中$S(x)=a_0\phi_0(x)+\dots + a_n\phi_n(x)$,这就是线性最小二乘曲线拟合

\begin{equation*}
    F(a_0, a_1, \dots, n) = \sum^m_{j=0}\rho(x_j)[S(x_j)-f_i]^2 = \sum^m_{j=0}\rho(x_j)[\sum^n_{i=0}a_i\phi_i(x_j)-f_j]^2
\end{equation*}
令$\frac{\partial F}{\partial a_i} = 0, i=0,1,\dots, n$则有:
\begin{equation}
    \frac{\partial F}{\partial a_k} = 2\sum^m_{j=0}\rho(x_j)[\sum^n_{i=0}a_i\phi_i(x_j)-f_j]\phi_k(x_j) = 0  
    \label{eq:2.2}
\end{equation}

\noindent 定义$(\phi_k, \phi_i) = \sum^m_{j=0}\rho(x_j)\phi_k(x_j)\phi_i(x_j) $为$\phi_k(x_j)$和$\phi_i(x_j)$在离散点集$X = \{x_0, x_1, \dots, x_m\}$上带权$\rho(x_j)$的内积,则\ref{eq:2.2}可写为:
\begin{equation}
    \sum^n_{i=0}(\phi_k, \phi_i)a_i = d_k, k = 0,1,\dots, n
    \label{eq:2.3}
\end{equation}
其中$d_k=(f, \phi_k)$,称\ref{eq:2.3}为法方程,或写成如下矩阵形式:
$$Ga=d$$
其中$a=(a_0, a_1, \dots, a_n)^T$,$d=(d_0, d_1, \dots, d_n)^T$,
$$G = \left[
    \begin{matrix}
        (\phi_0, \phi_0) & (\phi_0, \phi_1) & \dots & (\phi_0, \phi_n) \\
        (\phi_1, \phi_0) & (\phi_1, \phi_1) & \dots & (\phi_1, \phi_n) \\
        \vdots & \dots & \dots & \vdots \\
        (\phi_n, \phi_0) & (\phi_n, \phi_1) & \dots & (\phi_n, \phi_n)
    \end{matrix}
    \right]$$

\begin{definition}
    设$\phi_0, \phi_1, \dots, \phi_n\in [a,b]$的任意线性组合在$X=\{x_0, x_1, \dots, x_m\}(m\ge n)$上至多只有n个不同的零点,则称
    $\phi_0, \phi_1, \dots, \phi_n$在X上满足Haar条件
\end{definition}

\begin{theorem}
    若$\phi_0, \phi_1, \dots, \phi_n$在$X=\{x_0, x_1, \dots, x_m\}$上满足Harr条件,则$\phi_0, \phi_1, \dots, \phi_n$在x上线性无关,且法方程\ref{eq:2.3}的系数矩阵G非奇异。
\end{theorem}

\subsection{非线性最小二乘拟合}
\begin{example}
    \textbf{(重点)}若已知$(x_i, f_i), i=0,1,\dots, m$,且其分布近似于指数曲线,则可考虑用指数函数$S=be^{ax}$去拟合数据。
    
    为此,首先给出$F(a,b)=\sum^m_{i=0} (f_i-be^{ax_i})^2 $
    则可令$\frac{\partial F}{\partial a} = \frac{\partial F}{\partial b}=0, \Rightarrow \left\{
    \begin{array}{lr}
        \sum^m_{j=0}[f_j-be^{ax_j}]e^{ax_j}=0 \\
        \sum^m_{j=0}[f_j-be^{ax_j}]be^{ax_j}=0
    \end{array}    
     \right. $
     
     由 $S=be^{ax} $可得:$lnS(x)=lnb+ax $令:$z=lnS(x), \beta=lnb$则$z=\beta + ax$,选取$\Phi = \spn\{1, x\}, i.e. \phi_0(x)=1, \phi_1(x)=x$

     已知 $x = \{1, 1.25, 1.5, 1.75, 2\}, f = \{5.1, 5.79, 6.53, 7.45, 8.46\}$,则$lnf =\{1.629, 1.756, 1.876, 2.008, 2.135\} $ 则
     $(\phi_0, \phi_0) = 5, (\phi_0, \phi_1) = 7.5, (\phi_1, \phi_1) = 11.875, d_0 = (\phi_0, lnf) = 9.404, d_1=(\phi_1, lnf) = 14.424$

     则有如下法方程:
     $$\left[
         \begin{matrix}
             5 & 7.5 \\
             7.5 & 11.875 
         \end{matrix}
         \right]\left[
             \begin{matrix}
                 \beta \\ \alpha /
             \end{matrix}
             \right]=
             \left[
                 \begin{matrix}
                     9.404 \\ 14.424
                 \end{matrix}
                 \right]$$
    
    得到:$\beta = 1.1214, \alpha = 0.5064 \Rightarrow b = e^\beta \approx 3.069$,$\therefore S^*(x)=3.069e^{0.5064x} $
\end{example}

其他形式:$S(x) = \frac{1}{a_0+a_1x}$可转换为:$\frac{1}{S(x)} = a_0+a_1x$

\subsection{正交多项式最小二乘拟合}
如果能选取$\phi_0, \phi_1, \dots, \phi_n$关于点集$X=\{x_0, x_1, \dots, x_m\}$上带数$w_i, i=0,1,\dots, m$正交的函数族,i.e.$(\phi_j, \phi_k)=\sum^m_{i=0}w_i\phi_j(x_j)\phi_k(x_i)=\left\{
    \begin{array}{lr}
        0, i\neq k \\
        A_j > 0 , k=j
    \end{array}
    \right. $,则法方程变为:
    $$(\phi_k, \phi_k)a_k = d_k, k=0,1,\dots, n$$
    $$\Rightarrow a^*_k=(f,\phi_k)/(\phi_k, \phi_k) = \sum^m_{j=0}w_jf_j\phi_k(x_j)/\sum^m_{j=0}w_j\phi_k^2(x_j) $$

\section{插值}
\begin{definition}
    插值:就是寻找一个便于计算的简单函数$\phi(x)$使得其满足$\phi(x_j)=f(x_j), j=0,1,\dots, n$并以$\phi(x)$作为$f(x)$的近似值,称$\phi(x)$为插值函数,$f(x)$为被插函数。
\end{definition}

\subsection{Lagrange插值}
在多项式函数空间$P_n=\spn\{1,x,\dots, x^n\}$中寻找某一插值函数$L_n(x) $满足下列插值条件:
\begin{equation}
    L_n(x_j)=f(x_j), i=0,1,\dots, n 
    \label{eq:*}
\end{equation}
其中$L_n(x) = a_0 + a_1x + \dots+ a_nx^n$,由\ref{eq:*}式可以确定插值函数:$L_n(x)=a_0^* + a_1^*x + \dots+ a_n^*x^n $,其中$a_i^*(i=0,1,\dots, n)$为\ref{eq:*}式的解。

构造给出满足\ref{eq:*}的$L_n(x) $,选下列基函数:
\begin{equation}
    \begin{split}
        &l_i(x) = \frac{(x-x_0)\cdots(x-x_{i-1})(x-x_{i+1})\cdots(x-x_n)}{(x_i-x_0)\cdots(x_i-x_{i-1})(x_i-x_{i+1})\cdots(x_i-x_n)}\\
    =&\prod^n_{\substack{j=0\\j\neq i}}\frac{x-x_j}{x_i-x_j}
    \end{split}
\end{equation}
其满足:$l_i(x_k) = \left\{\begin{array}{lr}
    0, i\neq k \\ 1, i= k
\end{array}\right.$
则令:
\begin{equation}
    L_n(x) = \sum^n_{i=0}l_i(x)f(x_i)
\end{equation}
于是可证:$L_n(x_j)=f(x_j), j=0,1,\dots, n $
称$R(x)=f(x)-L_n(x) $为插值多项式的余项($R(x_j)=0$)。插值余项:
$$R(x) = f(x)-L_n(x)=\frac{f^{(n+1)}(\xi)}{(n+1)!}w_{n+1}(x) $$
其中$\xi \in (a,b), w_{n+1}(x)=\prod^n_{j=0}(x-x_j) , \Rightarrow f(x)=L_n(x)+R(x) $,$[a,b]$为插值区间。

\subsection{Newton插值}
若选取基函数$\phi_0(x)=1, \phi_1(x)=(x-x_0), \dots, \phi_k(x)=(x-x_0)(x-x_1)\dots(x-x_k) $,则Newton插值多项式可写为:$P(x)=a_0+a_1(x-x_0)+\dots+a_n(x-x_0)(x-x_1)\cdots(x-x_{n-1}) $,则由插值条件:
$P(x_j)=f(x_j),j=0,1,\dots, n $即可确定$a_j, j=0,1,\dots, n, i.e. a_0=P(x_0)=f(x_0) $.
\begin{align*}
    &a_0 = P(x_0) = f(x_0)\\
    &P(x_1)=a_0+a_1(x_1-x_0) = f(x_1) \\
    &P(x_2)=a_0+a_1(x_2-x_0)+a_n(x_2-x_0)(x_2-x_1) =f(x_2) \\
    &\dots \\
    &P(x_n)=a_0+a_1(x_n-x_0)+\dots+a_n(x_n-x_0)(x_n-x_1)\cdots(x_n-x_{n-1}) 
\end{align*}
用差商表示:
\begin{align*}
    &a_0=f(x_0), a_1 = \frac{f(x_1)-f(x_0)}{x_1-x_0}=f[x_0, x_1]  \\
    &a_2 = \frac{f[x_0, x_2]-f[x_0, x_1]}{x_2-x_1} = f[x_0, x_1, x_2] \\
    &\dots \\
    &a_k = f[x_0, x_1, \dots, x_k]
\end{align*}
由此可得Newton插值多项式:
\begin{equation}
    \rho(x) = f(x_0)+f[x_0,x_1](x-x_0) + \dots + f[x_0, x_1, \dots, x_n](x-x_0)\cdots(x-x_{n-1})
\end{equation}
注:Newton法的优势在于若新增节点,只需要计算新增的一项。

\begin{align*}
    R(x) = f(x)-P(x) \\
    H_n = \spn\{\phi_0, \dots, \phi_n\}
\end{align*}

\subsection{三次样条插值}
$$
R(x) = f(x)-L_n(x) = \frac{f^{(n+1)}(\xi)}{(n+1)!}w_{n+1}(x)
$$
若$f(x)=\frac{1}{1+x^2}$:随着插值节点增加,R(x)越来越大。

龙格现象?高次插值中易出现。随着节点增加无法改进。因此需要低次插值方法。

\begin{definition}
    设$[a,b]$上一个剖分:$\Delta: a=x_0<x_1<\dots<x_n=b$如果函数$S(x)$满足下列条件:
    \begin{enumerate}
        \item $S(x) \in C^n[a,b] $
        \item $S(x)$在每个子区间$[x_{i-1}, x_i],i=1,2,\dots, n$是3次多项式
    \end{enumerate}
    则称$S(x)$是关于节点剖分$\Delta$的三次样条函数
    
    若再给定$f(x)\in [a,b] $在节点上的值$f(x_j) $并使得$S(x_j)=f(x_j),j=0,1,\dots,n $则称$S(x) $为$f(x)$的三次样条插值函数。
\end{definition}

$S(x): [x_0, x_1][x_1,x_2],\dots, [x_{n-1},x_n] $共n段,共需确定4n个参数。\\
由$S(x)\in C^2(a,b)$有:
\begin{equation*}
    \left\{
    \begin{array}{lr}
        S(x_j-0) = S(x_j+0) \\
        S^{'}(x_j-0) = S^{'}(x_j+0) \\
        S^{{'}{'}}(x_j-0) = S^{{'}{'}}(x_j+0)
    \end{array}\right.
\end{equation*}
由插值条件共有$4n-2$个条件。再利用边界条件可得4n个方程。

边界条件的三种类型:
\begin{itemize}
    \item I型边界条件 $S^{'}(x_0)=f^{'}(x_0), S^{'}(x_n)=f^{'}(x_n) $
    \item II型边界条件 $S^{{'}{'}}(x_0)=f^{{'}{'}}(x_0), S^{{'}{'}}(x_n)=f^{{'}{'}}(x_n) $\\
        或$S^{{'}{'}}(x_0)=0, S^{{'}{'}}(x_n)=0 $,称为自然边界条件。
    \item III型边界条件(周期样条函数条件)\\
        $S^{(j)}(x_0)=S^{(j)}(x_n) $
\end{itemize}

考:两阶情况下,满足某种边界条件下的三次样条插值函数(给定部分参数)。

\section{数值积分}

定积分$\int^b_a f(x)dx=F(b)-F(a) $若无法积分,积分无意义?或者只有离散数据,不知道表达式,则需要定积分:
\begin{equation}
    I(f) = \int^b_af(x)dx = \lim_{\max(\Delta x)\rightarrow 0}\sum^n_{i=0}f(\xi_i)\Delta x_i
\end{equation}
数值积分则为选取不同$\xi_i$,来使得$\sum^n_{i=0}f(\xi_i)\Delta_i \approx \int^b_af(x)dx $和真实积分之间误差尽可能小。
i.e.$I(f)=\int^b_af(x)dx\approx \sum^n_{i=0}A_kf(x_k)$记为$I_n(f)$,其中$A_k$为求积系数,$x_k$为节点。(方法围绕如何选择$A_k, x_k$)

\subsection{插值型求积公式}
若已知节点$x_j$及其函数值$f(x_j),i=0,1,\dots, n$则可获得Lagrange插值多项式$L_n(x)=\sum^n_{k=0}l_k(x)f(x_k)$。进一步有:
\begin{equation}
    \begin{split}
        I(f) \approx I_n(f) &= \int^b_aL_n(x)dx = \sum^n_{k=0}[\int^b_al_k(x)dx]f(x_k) \\
        &= \sum^n_{k=0}A_kf(x_k)
    \end{split}
\end{equation}
其中$A_k = \int^b_al_k(x)dx$\\
误差:
\begin{equation*}
    \begin{split}
        R_n(f)&=\int^b_a[f(x)-L_n(x)]dx \\
        &=\int^b_a\frac{f^{(n+1)}(\xi)}{(n+1)!}w_{n+1}(x)dx
    \end{split}
\end{equation*}
求积公式至少n阶精度。$\because $若$f(x)$为n次多项式,$f^{(n+1)}=0$,$\therefore $至少n阶精度。

\begin{align}
    &\int^b_a f(x)dx \approx \sum^n_{k=0}A_kf(x_k) \label{eq:4.1} \\
    &R_n(f) = I(f)-I_n(f) = \int^b_a\frac{f^{(n+1)}(\xi)}{(n+1)!}w_{n+1}(x)dx
\end{align}

\subsection{求积公式的代数精度\textbf{(考点!!)}}

\begin{definition}
    \textbf{重点概念!}如果$R_n(x^m)=0, (m=0,1,\dots, k)$,则称求积公式\ref{eq:4.1}至少具有k次代数精度,如果$\exists R(x^{k+1})\neq 0$则称\ref{eq:4.1}
    的代数精度恰为k,或若当$f(x)$为任意次数$\le k$的多项式,求积公式
    $$\int^b_af(x)dx = \sum^n_{k=0}A_kf(x_k)$$精确成立,而对于某个$k+1$次多项式,公式不精确成立,则称求积公式具有k次代数精度
\end{definition}

注:\ref{eq:4.1}的代数精度至少为n,但最多只能到$n+1$(有n+1个节点,精度上限受限于节点已给定)如果节点和插值函数均可变($A_k, x_k$),则可提高精度,即高斯求积。

\subsection{Gauss型求积公式}
\begin{equation}
    \int^b_a\rho(x)f(x)dx \approx \sum^n_{k=0}A_kf(x_k)
    \label{eq:4.2}
\end{equation}

\begin{definition}
    如果一组节点$x_1, x_2, \dots, x_n \in [a,b]$能使求积公式\ref{eq:4.2}具有$2n-1$次代数精度,则称\ref{eq:4.2}为带权$\rho(x)$的Gauss型求积公式,其中
    $x_1, x_2, \dots, x_n$称为Gauss点(n个节点)
\end{definition}

\begin{example}
    试确定$A_1, A_2$和$x_1, x_2$使得下列求积公式$\int^1_{-1}f(x)dx\approx A_1f(x_1)+A_2f(x_2) $的代数精度尽可能高。
    
    解:选取$f(x)=1,x,x^2,x^3$则根据Gauss型求积公式定义可得
    \begin{equation*}
        \left\{
            \begin{array}{lr}
                A_1+A_2 = 2 = \int^1_{-1}1dx \\
                A_1x_1 + A_2x_2 = 0 = \int^1_{-1}xdx \\
                A_1x_1^2 + A_2x_2^2 = \frac{2}{3} = \int^1_{-1}x^2dx \\
                A_1x_1^3 + A_2x_2^3 = 0 = \int^1_{-1}x^3dx
            \end{array}
            \right.
    \end{equation*}

    解得:$\left\{\begin{array}{lr}
        x_1 = -\frac{1}{\sqrt{3}} \\
        x_2 = \frac{1}{\sqrt{3}}
    \end{array}\right. \left\{\begin{array}{lr}
        A_1 = 1 \\
        A_2 = 1
    \end{array}\right.$ 

    $\therefore \int^1_{-1}=f(x)dx \approx 1f(-\frac{1}{\sqrt{3}})+1f(\frac{1}{\sqrt{3}}) $

    $\int^1_{-1}x^4dx = \frac{2}{5} \neq \frac{2}{9} \therefore$该公式精度达不到4,为3.
\end{example}

\begin{theorem}
    求积公式\ref{eq:4.1}的节点是Gauss点的充要条件是函数$w_n(x)=\prod^n_{i=1}(x-x_i) $与任意次数不超过$n-1$的多项式$p(x)$在$[a,b]$上关于权$\rho(x)$正交.
    i.e. 
    $$
    \int^b_ap(x)w_n(x)\rho(x)dx = 0
    $$
\end{theorem}

\begin{proof}
    \begin{itemize}
        \item 必要性:对任何不超过$2n-1$次的多项式,\ref{eq:4.2}精确成立。$\forall p(x) \in H_{n-1}$,令$f(x)=p(x)w_{n}(x)\in H_{2n-1} $
        \item 充分性:要证$x_k$为高斯点。$\forall f(x)\in H_{2n-1}, f(x) = p(x)w_n(x)+q(x) $,其中$p(x), q(x)\in H_{n-1}$,第一项为0,第二项积分$q(x)$满足插值求积公式,精确成立。
    \end{itemize}
\end{proof}

Gauss型求积公式的构造过程:
\begin{itemize}
    \item 构造正交多项式得到高斯点。
    \item 确定求积系数$A_k$,
        可选取$l_k(x)=\prod^n_{\substack{i=1\\i\neq k}}\frac{x-x_i}{x_k-x_i} $($n-1$次多项式)\\
        $\therefore \int^b_a\rho(x)l_k(x)dx=\sum^n_{i=1}A_il_k(x_i)=A_k, (l_k(x_i)=0, i\neq k)$
\end{itemize}

Gauss型求积公式的误差:
\begin{equation*}
    \begin{split}
        R(f)&=\int^b_a\rho(x)f(x)dx - \sum^n_{k=1}A_kf(x_k) \\
        &= \frac{f^{(2n)}(\xi)}{(2n)!}\int^b_a\rho(x)w_n^2(x)dx, \xi \in (a,b)
    \end{split}
\end{equation*}

可以证明n点的求积公式最多只能达到$2n-1$阶代数精度。

下面给出几种特殊的高斯求积公式,即使用不同方法选取高斯点。

\subsubsection{Gauss-Legendre求积公式}
如果选取$[-1,1]$上Legendre正交多项式的零点作为Gauss点,即可得Gauss-Legendre求积公式$\int^1_{-1}f(x)dx\approx \sum^n_{k=1}A_kf(x_k) $
\begin{align}
    &A_k=\int^1_{-1}l_k(x)dx, k=1,2,\dots, n \\
    &R(f) = \frac{1}{[(2n)!]^3}\frac{2^{2n+1}}{2n+1}(n!)^4f^{(2n)}(\eta), \eta \in (-1,1)
\end{align}
若不是$(-1,1)$区间,而是任意$(a,b)$区间,做积分变量的坐标变换后求解$x=\frac{1}{2}(a+b)+\frac{1}{2}(b-a)t, t\in(-1,1)$

\subsubsection{Gauss-Chebyshev求积公式}
如果以$\rho(x)=\frac{1}{\sqrt{1-x^2}},[-1,1]$上Chebyshev多项式的零点作为Gauss点,即可构造Gauss-Chebyshev求积公式:
\begin{align*}
    &\int^1_{-1}\frac{1}{\sqrt{1-x^2}}f(x)dx \approx \sum^n_{k=1}A_kf(x_k) \\
    &A_k=\frac{\pi}{n} \\
    &x_k=cos\frac{2k-1}{2n}\pi, k=1,2,\dots,n\\
    &R(f) = \frac{\pi}{2^{n-1}(2n)!}f^{(2n)}(\eta), \eta \in(-1,1)
\end{align*}

\begin{example}
    \textbf{重要!!}构造具有下列形式的Gauss型求积公式:
    \begin{equation}
        \int^1_0\sqrt{x}f(x)dx\approx A_1f(x_1) + A_2f(x_2)
        \label{eq:ex}
    \end{equation}
    \begin{itemize}
        \item 方法一:待定系数法:令$f(x)=1,x,x^2,x^3$分别代入\ref{eq:ex}使其精度成立即可确定$A_1, A_2, x_1,x_2$
        \item 方法二:二次正交多项式构造\\
            取:$\rho(x) = \sqrt{x}$,则构造二次正交多项式如下:$\phi_0(x)=1, \phi_1(x)=x-\frac{3}{5}$
            $$\because \alpha_1 = \frac{(x\phi_0, \phi_0)}{(\phi_0, \phi_0)} = \frac{\int^1_0x^{3/2}dx}{\int^1_0x^{1/2}dx}=\frac{3}{5}$$
            $$\phi_2(x) = (x-\alpha_2)\phi_1(x)- \beta_2\phi_0(x) = x^2-\frac{10}{9}x+\frac{5}{21}$$
            $$\alpha_2 = \frac{(x\phi_1, \phi_1)}{(\phi_1, \phi_1)}=\frac{\int^1_0x^{3/2}(x-\frac{3}{5})^2dx}{\int^1_0x^{1/2}(x-\frac{3}{5})^2dx}=\frac{23}{45} $$
            $$\beta_2 = \frac{(\phi_1, \phi_1)}{(\phi_0, \phi_0)}=\frac{\int^1_0x^{1/2}(x-\frac{3}{5})^2dx}{\int^1_0x^{1/2}dx}=\frac{12}{175}$$

            令$\phi_2(x)=0$,则$x_1=0.821162, x_2=0.289949, A_1=\int^1_0\sqrt{x}l_1(x)dx = \int^1_0\sqrt{x}\frac{x-x_2}{x_1-x_2}\approx 0.38911$
            $A_2=\int^1_0\sqrt{x}l_2(x)dx \approx 0.277556$
            $$\therefore \int^1_0\sqrt{x}f(x)dx \approx 0.38911f(x_1)+0.277556f(x_2) $$
        \item 方法三:取$\rho(x)=\sqrt{x}$,则构造二次正交多项式如下,选取$\phi_0=1, \phi_1(x)=x+a, \phi_2(x)=x^2+bx+c$,则利用正交性确定$\phi_1, \phi_2$,i.e.
            $$
            \left\{
            \begin{array}{lr}
                \int^1_0\sqrt{x}(x+a)dx=0 \\
                \int^1_0\sqrt{x}(x^2+bx+c)dx=0 \\
                \int^1_0\sqrt{x}(x+a)(x^2+bx+c)dx=0
            \end{array}\right.
            $$
    \end{itemize}
\end{example}

\end{document}
